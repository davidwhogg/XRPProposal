\documentclass[12pt]{article}
\setlength{\headheight}{2ex} % must be *before* \input{hogg_nasa}
\usepackage{amsmath}
% \usepackage{physics}
\setlength{\headsep}{3ex} % must be *before* \input{hogg_nasa}
\input{hogg_nasa}
\pagestyle{myheadings}
\markboth{}{\color{gray}\sffamily Hogg \& Daunt / New tools for extreme-precision spectrographs}
\begin{document}

\section{Team expertise}

Hogg...

The PI has a strong reputation for open science, performing data analysis and writing in the open, and a strong preference for working in open and public data sets. The PI is a long-term supporter and maintainer of open-source software systems of wide use in the astronomical community HOGG CITE THINGS

Daunt...

Also we will collaborate with Bedell... Add a strong statement about our work on workshops, and our support from FI.

\section{Prior NASA support}

\clearpage
\section{Biographical sketch --- David W Hogg}

\setlength{\tabcolsep}{0em}
\begin{tabular}{lll}
Center for Cosmology and Particle Physics & \hspace{6em} & \texttt{david.hogg@nyu.edu} \\
Department of Physics                     & & \url{http://cosmo.nyu.edu/hogg/} \\
New York University                       & & 
\end{tabular}

\paragraph{Education}
\begin{list}{}{\hogglist}
\item
PhD 1998, Physics, California Institute of Technology.
\item
SB 1992 (Physics), Massachusetts Institute of Technology.
\end{list}

\paragraph{Current positions}
\begin{list}{}{\hogglist}
\item
Professor of Physics and Data Science, New York University, 2014--.
\item
(part-time) Group Leader, Astronomical Data Group, Flatiron Institute, 2017--.
\end{list}

\paragraph{Recent service}
% reverse chronological by END date.
\begin{list}{}{\hogglist}
\item
\project{Sloan Digital Sky Survey IV} Collaboration Council,
2013--present.
\item
US National Astronomy and Astrophysics Advisory Committee (AAAC)
2014--2017.
\end{list}

\paragraph{Relevant management experience}
%%% reverse order by expxsyiration date
\begin{list}{}{\hogglist}
\item
Advisor for 11 graduated PhD students and 2 current PhD students:
Morad~Masjedi (2007);
Dustin~Lang (2009);
Ronin~Wu (2010);
Jo~Bovy (2011);
Adi~Zolotov (2011);
Tao~Jiang (2012);
Fengji~Hou (2014);
Daniel~Foreman-Mackey (2015);
Mohammadjavad~Vakili (2017);
Dun~Wang (2018);
Alex~Malz (2019);
Kate~Storey-Fisher (current);
Suroor~Gandhi (current).
\item
Mentor for postdocs at NYU, the Flatiron Institute, and MPIA.
\item
Member of the Oversight Committee for the \textsc{nasa} \project{Spitzer Space Telescope}.
\item
Member of the Technical Advisory Group for the \project{Sloan Digital Sky Survey V}.
\item
Co-developer and co-maintainer for many open-source code bases,
including:\\
\project{Astrometry.net} (automatic recognition and calibration of arbitrary astronomical imaging);\\
\project{The Joker} (a custom Monte Carlo sampler for exoplanets and binary stars);\\
\project{wobble} (a data-driven model to measure radial velocities at extreme precision);\\
and many more.
\item
PI on many past Federally funded projects including these recent
\textsc{nasa} projects:\\
\grantnumber{80NSSC19K0533}{Bean}: 
\textit{Improving the sensitivity of radial velocity spectrographs with
data-driven techniques},
\usd{308,326}, 2019--2021;\\
\grantnumber{NNX16AC70G}{Hogg}:
\textit{Ultra-precise photometry in crowded fields: A self-calibration approach},
\usd{100,000}, 2016--2017;\\
\grantnumber{NNX12AI50G}{Hogg}:
\textit{The Lives and Deaths of Planets and Stars in the Value-Added UV Photon Catalog},
\usd{473,705}, 2012--2017; and\\
\grantnumber{AR-13250}{Hogg}:
\textit{Probabilistic Self-Calibration of the \textsc{wfc3} IR Channel},
\usd{119,988}, 2013--2016.
\end{list}

\hypersetup{linkcolor=black}%
\paragraph{Selected relevant publications}
\begin{list}{}{\hogglist}
\item
Hogg,~D.~W., Myers,~A.~D., \& Bovy,~J., 2010,
\doi{10.1088/0004-637X/725/2/2166}{Inferring the eccentricity distribution},
\textit{Astrophys.\,J.}\ \textbf{725} 2166--2175.
\item
Hou,~F., Goodman,~J., Hogg,~D.~W., Weare,~J., \& Schwab,~C., 2012,
\doi{10.1088/0004-637X/745/2/198}{An affine-invariant sampler for exoplanet fitting and discovery in radial velocity data},
\textit{Astrophys.\,J.}\ \textbf{745} 198.
\item
Tsalmantza,~P. \& Hogg,~D.~W., 2012,
\doi{10.1088/0004-637X/753/2/122}{A data-driven model for spectra:\ Finding double redshifts in the \project{Sloan Digital Sky Survey}},
\textit{Astrophys.\,J.}\ \textbf{753} 122.
\item
Foreman-Mackey,~D., Montet,~B.~T., Hogg,~D.~W., Morton,~T.~D.,
Wang,~D., \& Sch\"olkopf,~B., 2015,
\doi{10.1088/0004-637X/806/2/215}{A systematic search for transiting planets in the \project{K2} data},
\textit{Astrophys.\,J.}\ \textbf{806} 215.
\item
Ness,~M., Hogg,~D.~W., Rix,~H.-W., Ho,~A.~Y.~Q., \& Zasowski,~G., 2015,
\doi{10.1088/0004-637X/808/1/16}{\project{The~Cannon}: A data-driven
approach to stellar label determination},
\textit{Astrophys.\,J.}\ \textbf{808} 16.
\item
Price-Whelan,~A.~M., Hogg,~D.~W., Foreman-Mackey,~D., \& Rix,~H.-W., 2017,
\doi{10.3847/1538-4357/aa5e50}{The Joker: A Custom Monte Carlo Sampler for Binary-star and Exoplanet Radial Velocity Data},
\textit{Astrophys.\,J.}\ \textbf{837} 20.
\item
Anderson,~L., Hogg~D.~W., Leistedt,~B., Price-Whelan,~A.~M., \& Bovy,~J., 2018,
\doi{10.3847/1538-3881/aad7bf}{Improving \project{Gaia} parallax precision with a data-driven model of stars},
\textit{Astron.\,J.}\ \textbf{156} 145.
\item
El-Badry,~K., Ting,~Y.-S., Rix,~H.-W., Quataert,~E., Weisz,~D.~R., Cargile,~P., Conroy,~C., Hogg,~D.~W., Bergemann,~M., \& Liu,~C., 2018,
\doi{10.1093/mnras/sty240}{Discovery and Characterization of 3000+ Main-Sequence Binaries from \project{APOGEE} Spectra},
\textit{Mon.\,Not.\,R.\,Astr.\,Soc.}\ \textbf{476} 528--553.
\item
Bedell,~M., Hogg,~D.~W., Foreman-Mackey,~D., Montet,~B.~T., \& Luger,~R.,
\arxiv{1901.00503}{Wobble: A data-driven method for precision radial velocities}.
\end{list}

\paragraph{Other important publications}
\begin{list}{}{\hogglist}
\item
Hogg,~D.~W. \etal, 2004,
\doi{10.1086/381749}{The dependence on environment of the color--magnitude relation of galaxies},
\textit{Astrophys.\,J.\,Lett.}\ \textbf{601} L29--L32.
\item
Willman,~B., Blanton,~M.~R., West,~A.~A., Dalcanton,~J.~J, Hogg,~D.~W., Schneider,~D.~P., Wherry,~N., Yanny,~B., \& Brinkmann,~J., 2005,
\doi{10.1086/430214}{A new Milky Way companion:\ Unusual globular cluster or extreme dwarf satellite?},
\textit{Astron.\,J.}\ \textbf{129} 2692--2700.
\item
Eisenstein,~D.~J., Zehavi,~I., Hogg,~D.~W., \etal, 2005,
\doi{10.1086/466512}{Detection of the baryon acoustic peak in the large-scale correlation function of \project{Sloan Digital Sky Survey} Luminous Red Galaxies},
\textit{Astrophys.\,J.}\ \textbf{633} 560--574.
\item
Lang,~D., Hogg,~D.~W., Mierle,~K., Blanton,~M., \& Roweis,~S., 2010,
\doi{10.1088/0004-6256/139/5/1782}{\project{Astrometry.net}:\ Blind astrometric calibration of arbitrary astronomical images},
\textit{Astron.\,J.}\ \textbf{139} 1782--1800.
\item
Bovy,~J., Hennawi,~J.~F., Hogg,~D.~W., \etal, 2011,
\doi{10.1088/0004-637X/729/2/141}{Think outside the color-box:\ Probabilistic target selection and the \textsc{sdss-xdqso} quasar targeting catalog},
\textit{Astrophys.\,J.}\ \textbf{729} 141.
\item
Oppenheimer,~B.~R. \etal, 2013,
\doi{10.1088/0004-637X/768/1/24}{Reconnaissance of the HR~8799 exosolar system.\ I.\ Near-infrared spectroscopy},
\textit{Astrophys.\,J.}\ \textbf{768} 24.
\item
Foreman-Mackey,~D., Hogg,~D.~W., Lang,~D., \& Goodman,~J., 2013,
\doi{10.1086/670067}{\project{emcee}:\ The MCMC Hammer},
\textit{Pubs.\,Astr.\,Soc.\,Pac.}\ \textbf{125} 306--312.
\label{rpcount}\end{list}

\clearpage
\section{Biographical sketch --- Matthew Daunt}

\clearpage
\section{Current and pending support}

\clearpage
\section{Budget narrative}

\clearpage
\section{Work effort}

Needs to be in tabular form?

\clearpage
\section{Facilities and equipment}

The NYU Department of Physics is part of the NYU Faculty of Arts \&
Science and includes the Center for Cosmology and Particle Physics,
the Center for Soft Matter Research, and the Center for Quantum Phenomena.
A wide variety of sponsored
research activities take place in the Department.  These range from
the theoretical to the pragmatic, and include a broad spectrum of
interactions with such disciplines as biology, medicine, chemistry,
applied mathematics, and computer science.

In addition to benefitting from this research activity directly and
indirectly, as a member of the Department, the PI receives through the
Department staff support for clerical work, post-award grant support,
and for computing (expanded upon below).

\paragraph{Computing Equipment:}
The astrophysics group at NYU maintains some high-performance computers
for the PI, and substantial storage machines.
The latter contain more than 100~Tb of disk space, most of which is filled
with astrophysical imaging data.  The Physics Department maintains a
state-of-the-art controlled-environment computer room that houses and
protects the computers that will be used in this project.

\paragraph{General computer resources:}
The University has hired a Director of Scientific Computing for the
NYU Center for Cosmology and Particle Physics (the PI's home).  His
responsibilities include management of the cluster and the data
servers, and overall management and supervision of the Center's
computer system.  He will oversee the computer hardware used in this
project.

A variety of computing resources are available within the Physics
Department, including UNIX workstations, laptops, laser color
printers, etc.  The Department runs a network of several hundred
desktop workstations, file servers, in addition to the multi-processor
computational server mentioned above.

Computational needs are also supported through the University's
Academic Computing Services, a unit of the NYU-wide Information
Technology Services offering an additional wide range of computational
resources in support of research and instruction.  These include a
variety of computing platforms, including several high-performance
multi-CPU systems, and scientific software.  Consultants are available
to assist in the use of these resources.

Finally, the NYU Center for Data Science (where the PI has an appointment)
maintains some specialized high-performance machines, including a GPU
cluster for machine-learning
applications. These resources could be relevant to the proposed project,
especially as some development might be optimized for GPUs.

\paragraph{Office space:}
The offices of the PI and Co-I will be located in the NYU Physics
Department.

\paragraph{Library resources:}
In addition to an enormous book collection, the NYU libraries hold
current subscriptions to hundreds of hardcopy and electronic journals.
Libraries provides access to all journals relevant to this project and such
databases as MathSciNet, the Web of Science (science citation index),
and the ACM Digital Library. They also provide resources to support
researchers with preservation of data, code, and models.

\paragraph{Meeting spaces:}
The Center for Cosmology and Particle Physics, the NYU Libraries, the
NYU Center for Data Science, and the NYU Kimmel Center for Student
Life all have spaces that are available to the PI for meetings and
workshops.

\paragraph{Experimental and hardware facilities:}
The Department has large experimental facilities, including machining,
imaging, and clean facilities, but none of these are directly relevant
to this project, except insofar as they are part of the rich
intellectual and research atmosphere.

\clearpage
\section{Letters}

\end{document}